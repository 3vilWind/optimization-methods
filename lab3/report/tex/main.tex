\documentclass[a4paper,12pt]{article}

\usepackage{cmap}					
\usepackage[T2A]{fontenc}
\usepackage[utf8]{inputenc}
\usepackage[english,russian]{babel}
\usepackage{hyperref}
\usepackage{graphicx}
\usepackage[table,xcdraw]{xcolor}

\hypersetup{
    colorlinks=true,
    linkcolor=blue,
    filecolor=magenta,      
    urlcolor=cyan,
}

\usepackage{amsmath,amsfonts,amssymb,mathtools}
\usepackage{icomma}

\usepackage{euscript}
\usepackage{mathrsfs}
\usepackage{graphicx}
\usepackage{eso-pic}

\begin{document}

\begin{titlepage}

\thispagestyle{empty}

\centerline{НИУ ИТМО}
\centerline{Факультет Информационных Технологий и Программирования}
\centerline{Направление "Прикладная Математика и Информатика"}

\vfill

\centerline{\huge{Лабораторная работа 3}}
\centerline{\large{курса ``Методы Оптимизации'' }}
\vspace{1cm} 
\centerline{\large{Выполнили Раков Николай, Булкина Милена}}
\vfill

\centerline{Санкт-Петербург, 2021}
\clearpage
\end{titlepage}

%\title{Методы Оптимизации. Лабораторная работа №3}
%\author{Раков Николай, Булкина Милена}
%\date{}

\section{Постановка задания}
\begin{enumerate}
\item Реализовать прямой метод решения СЛАУ на основе LU-разложения с учетом следующих требований:
\begin{itemize}
\item формат матрицы – профильный;
\item размерность матрицы, элементы матрицы и вектор правой части читать из файлов, результаты записывать в файл;
\item в программе резервировать объём памяти, необходимый для хранения в нем только одной матрицы и необходимого числа векторов (то есть треугольные матрицы, полученные в результате разложения, должны храниться на месте исходной матрицы);
\item элементы матрицы обрабатывать в порядке, соответствующем формату хранения, то есть необходимо работать именно со столбцами верхнего и строками нижнего треугольников.
\end{itemize}

\item Провести исследование реализованного метода на матрицах, число
обусловленности которых регулируется за счёт изменения диагонального преобладания (то есть оценить влияние увеличения числа обусловленности на точность решения).
\begin{itemize}
\item Оценить погрешность решения для каждого k, для которого система вычислительно разрешима.
\item Для одного из значений k попытаться найти операцию, вызывающую скачкообразное накопление погрешности, пояснить полученные результаты.
\end{itemize}

\item Провести аналогичные исследования на матрицах Гильберта различной размерности. Матрица Гильберта размерности k строится следующим образом:
\begin{equation*}
    a_{ij} = \frac{1}{i + j - 1}, i,j = \overline{1,k}
\end{equation*}

\item Реализовать метод Гаусса с выбором ведущего элемента для плотных матриц. Сравнить метод Гаусса по точности получаемого решения и по количеству действий с реализованным прямым методом LU – разложения.

\item Реализовать метод сопряженных градиентов для решения СЛАУ, матрица которых хранится в разреженном строчно – столбцовом и является симметричной.

\end{enumerate}

\section{Формат хранения матриц}
\subsection{Плотный}
Плотный формат хранения матриц используется, когда матрица не обладает определенной структурой и имеет малые размеры. Этот формат требует $\mathcal{O}(n^2)$ памяти.

\subsection{Профильный}
Профильный формат хранения матриц используется, когда ненулевые элементы матрицы расположены в произвольном порядке, но при этом они сосредоточены у главной диагонали. 
Этот формат занимает меньше памяти, чем плотный, если ненулевые элементы плотно расположены около главной диагонали.

\noindentМассивы, необходимые для хранения профильной квадратной матрицы:
\begin{itemize}
    \item Массивы al и au - вещественные массивы, которые хранят внедиагональные элементы нижнего (по строкам) и верхнего (по столбцам) треугольника матрицы.
    \item Массив ia[n] хранит информацию о профиле. Он содержит указатели начала строк (столбцов) нижнего (верхнего) треугольника в массиве al (au). Индексный массив ia формируется следующим образом:
\begin{itemize}
    \item первые два элемента индексного массива ia всегда равны 1: ia[0] = ia[1] = 0, так как в первой строке в нижнем треугольнике нет внедиагональных элементов
    \item к элементу ia[k] добавляется количество элементов в профиле k -ой строки и получается элемент ia[k + 1].
\end{itemize}
    \item di[n] - вещественный массив. Он хранит все диагональные элементы.
\end{itemize}

\subsection{Разреженный строчно-столбцовый симметричный формат}
В отличие от профильного формата дополнительно необходим еще один массив, содержащий информацию о положении внедиагонального элемента в строке (для нижнего треугольника) или в столбце (для верхнего треугольника). Используется, когда количество нулевых элементов велико.

\noindent Необходимые массивы для данного формата:
\begin{itemize}
    \item Вещественные массивы al и au, которые хранят внедиагональные элементы нижнего (по строкам) и верхнего (по столбцам) треугольника матрицы.
    \item Целочисленный массив ja хранит номера столбцов (строк) внедиагональных элементов нижнего (верхнего) треугольника матрицы.
    \item Целочисленный массив ia[n] хранит указатели начала строк (столбцов) в массивах ja , al и au.
    \item di[n] - вещественный массив. Он хранит все диагональные элементы.
\end{itemize}

\section{Решение СЛАУ}
\begin{equation*} 
 \begin{cases}
   a_{11}x_1 + a_{12}x_2 \ldots +  a_{1n}x_n = b_1\\ 
   a_{21}x_1 + a_{22}x_2 \ldots +  a_{2n}x_n = b_2\\ 
   \ldots\\
   a_{n1}x_1 + a_{n2}x_2 \ldots +  a_{nn}x_n = b_n
 \end{cases} 
\end{equation*}
\subsection{Метод Гаусса с выбором главного элемента}
Данный метод имеет преимущество перед обычным методом последовательного исключения Гаусса, так как в последнем среди ведущих элементов могут оказаться очень маленькие по абсолютной величине. Тогда при делении на данные элементы получается большая вычислительная погрешность.

В данном методе выбирается наибольший по модулю элемент.

\subsection{LU -  разложение}

Разложение LU:
\begin{itemize}
    \item $L_{ii} = A_{ii} - \sum\limits_{k = 1}^{i - 1} L_{ik}U_{ki}$
    \item $U_{ii} = 1$
    \item $L_{ij} = A_{ij} - \sum\limits_{k = 1}^{j - 1} L_{ik}U_{kj}, j = 1\dots i - 1$
    \item $L_{ji} = \dfrac{1}{L_{jj}}\left(A_{ji} - \sum\limits_{k = 1}^{j - 1} L_{jk}U_{ki}, j = 1\dots i - 1)\right)$
\end{itemize}
Алгоритм решения СЛАУ:
\begin{enumerate}
    \item Находим L и U из матрицы A
    \item Прямым ходом Гаусса решаем Ly = b, находим y
    \item Обратным ходом решаем Ux = y, находим x
\end{enumerate}

\subsection{Использование МСГ для решения СЛАУ}

Можно использовать МСГ для решения СЛАУ с симметричной положительно определенной матрицей А. Нам нужно найти решение уравнение Ax - b = 0 - это градиент функции $f(x) = \frac{1}{2}\left(Ax, x\right) - \left(b, x\right)$, следовательно решение СЛАУ можно свести к минимизации f(x).

\underline{Алгоритм:}
\begin{enumerate}
    \item Подготовка перед итерационным процессом:
    \begin{itemize}
        \item Выберем начальное приближение $x^{0}$
        \item $r^{0} = b - Ax^{0}$
        \item $z^{0} = r^{0}$
    \end{itemize}
    \item k-ая итерация метода:
    \begin{itemize}
    \item ${\displaystyle \alpha _{k}={\frac {(r^{k-1},r^{k-1})}{(Az^{k-1},z^{k-1})}}}$
        \item $x^k = x^{k-1} + \alpha_k z^{k-1}$
        \item $r^k = r^{k-1} - \alpha_k A z^{k-1}$
        \item $\beta_k = \frac{(r^k, r^k)}{(r^{k-1}, r^{k-1})}$
        \item ${\displaystyle z^{k}=r^{k}+\beta _{k}z^{k-1}}$
    \end{itemize}
\end{enumerate}

\section{Результаты исследований}

\clearpage

\begin{table}[]
\centering
\caption{Гаусс на основе LU-разложения на матрицах Гильберта}
\begin{tabular}{|c|c|c|}
\hline
\rowcolor[HTML]{C9FB98} 
n     & $\|x^* - x_k\|$            & $\dfrac{\|x^* - x_k\|}{\|x^*\|}$           \\ \hline
\rowcolor[HTML]{EEFB98} 
3    & 2.625108e-14                                              & 7.015897e-15 \\ \hline
\rowcolor[HTML]{EEFB98} 
4    & 8.830921e-13                                              & 1.612298e-13 \\ \hline
\rowcolor[HTML]{EEFB98} 
5    & 3.558406e-11                                              & 4.798154e-12 \\ \hline
\rowcolor[HTML]{EEFB98} 
6    & 2.319962e-11                                              & 2.431981e-12 \\ \hline
\rowcolor[HTML]{EEFB98} 
7    & 2.009583e-08                                              & 1.698407e-09 \\ \hline
\rowcolor[HTML]{EEFB98} 
8    & 1.333407e-06                                              & 9.335716e-08 \\ \hline
\rowcolor[HTML]{EEFB98} 
9    & 7.270362e-05                                              & 4.306591e-06 \\ \hline
\rowcolor[HTML]{EEFB98} 
10   & 3.503464e-03                                              & 1.785531e-04 \\ \hline
\rowcolor[HTML]{EEFB98} 
20   & 8.003411e+02                                              & 1.493943e+01 \\ \hline
\rowcolor[HTML]{EEFB98} 
30   & 1.081093e+03                                              & 1.111814e+01 \\ \hline
\rowcolor[HTML]{EEFB98} 
40   & 5.560919e+03                                              & 3.737299e+01 \\ \hline
\rowcolor[HTML]{EEFB98} 
50   & 2.138525e+04                                              & 1.032189e+02 \\ \hline
\rowcolor[HTML]{EEFB98} 
100  & 6.160375e+04                                              & 1.059069e+02 \\ \hline
\rowcolor[HTML]{EEFB98} 
1000 & 3.386377e+08                                              & 1.853405e+04 \\ \hline
\end{tabular}
\end{table}

Матрицы Гильберта являются стандартным примером плохо обусловленных матриц, число обусловленности возрастает как $O((1+{\sqrt  {2}})^{{4n}}/{\sqrt{n}})$. Именно поэтому даже при небольших n невязка велика.



\clearpage

\begin{table}[]
\centering
\caption{Гаусс на основе LU-разложения на матрицах с диагональным преобладанием}
\begin{tabular}{|c|c|c|c|}
\hline
\rowcolor[HTML]{FFE5B4} 
n    & k  &   $\|x^* - x_k\|$           &       $\dfrac{\|x^* - x_k\|}{\|x^*\|}$       \\ \hline
\rowcolor[HTML]{FDEED9} 
10   & 0  & 1.306113e-13 & 6.656567e-15 \\ \hline
\rowcolor[HTML]{FDEED9} 
10   & 1  & 1.113850e-12 & 5.676703e-14 \\ \hline
\rowcolor[HTML]{FDEED9} 
10   & 2  & 1.357311e-11 & 6.917498e-13 \\ \hline
\rowcolor[HTML]{FDEED9} 
10   & 3  & 8.940309e-11 & 4.556404e-12 \\ \hline
\rowcolor[HTML]{FDEED9} 
10   & 4  & 1.227695e-09 & 6.256913e-11 \\ \hline
\rowcolor[HTML]{FDEED9} 
10   & 5  & 1.644495e-08 & 8.381121e-10 \\ \hline
\rowcolor[HTML]{FDEED9} 
10   & 6  & 8.972282e-08 & 4.572698e-09 \\ \hline
\rowcolor[HTML]{FDEED9} 
10   & 7  & 1.401666e-06 & 7.143552e-08 \\ \hline
\rowcolor[HTML]{FDEED9} 
10   & 8  & 2.194633e-05 & 1.118488e-06 \\ \hline
\rowcolor[HTML]{FDEED9} 
10   & 9  & 8.997474e-05 & 4.585538e-06 \\ \hline
\rowcolor[HTML]{FDEED9} 
10   & 10 & 1.640537e-03 & 8.360949e-05 \\ \hline
\rowcolor[HTML]{FDEED9} 
100  & 0  & 6.364530e-10 & 1.094166e-12 \\ \hline
\rowcolor[HTML]{FDEED9} 
100  & 1  & 4.825382e-09 & 8.295615e-12 \\ \hline
\rowcolor[HTML]{FDEED9} 
100  & 2  & 3.058891e-08 & 5.258730e-11 \\ \hline
\rowcolor[HTML]{FDEED9} 
100  & 3  & 2.108344e-07 & 3.624585e-10 \\ \hline
\rowcolor[HTML]{FDEED9} 
100  & 4  & 2.877851e-06 & 4.947493e-09 \\ \hline
\rowcolor[HTML]{FDEED9} 
100  & 5  & 5.980700e-06 & 1.028179e-08 \\ \hline
\rowcolor[HTML]{FDEED9} 
100  & 6  & 1.293710e-04 & 2.224098e-07 \\ \hline
\rowcolor[HTML]{FDEED9} 
100  & 7  & 1.217430e-03 & 2.092960e-06 \\ \hline
\rowcolor[HTML]{FDEED9} 
100  & 8  & 2.897889e-02 & 4.981941e-05 \\ \hline
\rowcolor[HTML]{FDEED9} 
100  & 9  & 2.569926e-01 & 4.418120e-04 \\ \hline
\rowcolor[HTML]{FDEED9} 
100  & 10 & 2.680254e+00 & 4.607791e-03 \\ \hline
\rowcolor[HTML]{FDEED9} 
1000 & 0  & 6.124098e-07 & 3.351793e-11 \\ \hline
\rowcolor[HTML]{FDEED9} 
1000 & 1  & 8.628147e-06 & 4.722289e-10 \\ \hline
\rowcolor[HTML]{FDEED9} 
1000 & 2  & 9.231194e-05 & 5.052344e-09 \\ \hline
\rowcolor[HTML]{FDEED9} 
1000 & 3  & 1.615442e-03 & 8.841512e-08 \\ \hline
\rowcolor[HTML]{FDEED9} 
1000 & 4  & 7.150361e-03 & 3.913479e-07 \\ \hline
\rowcolor[HTML]{FDEED9} 
1000 & 5  & 2.549932e-01 & 1.395609e-05 \\ \hline
\rowcolor[HTML]{FDEED9} 
1000 & 6  & 7.899812e-01 & 4.323662e-05 \\ \hline
\rowcolor[HTML]{FDEED9} 
1000 & 7  & 1.701534e+01 & 9.312704e-04 \\ \hline
\rowcolor[HTML]{FDEED9} 
1000 & 8  & 2.542212e+01 & 1.391383e-03 \\ \hline
\rowcolor[HTML]{FDEED9} 
1000 & 9  & 1.187467e+03 & 6.499153e-02 \\ \hline
\rowcolor[HTML]{FDEED9} 
1000 & 10 & 9.924544e+03 & 5.431823e-01 \\ \hline
\end{tabular}
\end{table}

\clearpage

\begin{table}[]
\centering
\caption{Метод Гаусса с выбором ведущего элемента на матрицах с диагональным преобладанием}
\begin{tabular}{|c|c|c|c|}
\hline
\rowcolor[HTML]{FFE9F5} 
n    & k  &   $\|x^* - x_k\|$           &       $\dfrac{\|x^* - x_k\|}{\|x^*\|}$       \\ \hline
\rowcolor[HTML]{F1E9FF} 
10   & 0  & 1.323308e-13 & 6.744200e-15 \\ \hline
\rowcolor[HTML]{F1E9FF} 
10   & 1  & 5.093242e-13 & 2.595757e-14 \\ \hline
\rowcolor[HTML]{F1E9FF} 
10   & 2  & 1.452915e-11 & 7.404738e-13 \\ \hline
\rowcolor[HTML]{F1E9FF} 
10   & 3  & 1.785659e-10 & 9.100563e-12 \\ \hline
\rowcolor[HTML]{F1E9FF} 
10   & 4  & 7.609669e-11 & 3.878247e-12 \\ \hline
\rowcolor[HTML]{F1E9FF} 
10   & 5  & 2.750950e-08 & 1.402014e-09 \\ \hline
\rowcolor[HTML]{F1E9FF} 
10   & 6  & 2.090429e-07 & 1.065381e-08 \\ \hline
\rowcolor[HTML]{F1E9FF} 
10   & 7  & 3.747852e-07 & 1.910082e-08 \\ \hline
\rowcolor[HTML]{F1E9FF} 
10   & 8  & 3.320164e-06 & 1.692112e-07 \\ \hline
\rowcolor[HTML]{F1E9FF} 
10   & 9  & 3.660984e-05 & 1.865810e-06 \\ \hline
\rowcolor[HTML]{F1E9FF} 
10   & 10 & 5.416534e-04 & 2.760521e-05 \\ \hline
\rowcolor[HTML]{FFE9F5} 
100  & 0  & 6.072080e-10 & 1.043889e-12 \\ \hline
\rowcolor[HTML]{FFE9F5} 
100  & 1  & 2.574989e-09 & 4.426825e-12 \\ \hline
\rowcolor[HTML]{FFE9F5} 
100  & 2  & 4.272554e-08 & 7.345215e-11 \\ \hline
\rowcolor[HTML]{FFE9F5} 
100  & 3  & 2.570235e-07 & 4.418652e-10 \\ \hline
\rowcolor[HTML]{FFE9F5} 
100  & 4  & 3.726496e-06 & 6.406451e-09 \\ \hline
\rowcolor[HTML]{FFE9F5} 
100  & 5  & 2.684952e-05 & 4.615868e-08 \\ \hline
\rowcolor[HTML]{FFE9F5} 
100  & 6  & 4.180415e-04 & 7.186813e-07 \\ \hline
\rowcolor[HTML]{FFE9F5} 
100  & 7  & 6.500045e-03 & 1.117463e-05 \\ \hline
\rowcolor[HTML]{FFE9F5} 
100  & 8  & 3.351292e-02 & 5.761415e-05 \\ \hline
\rowcolor[HTML]{FFE9F5} 
100  & 9  & 2.643501e-01 & 4.544608e-04 \\ \hline
\rowcolor[HTML]{FFE9F5} 
100  & 10 & 4.798962e-01 & 8.250196e-04 \\ \hline
\rowcolor[HTML]{F1E9FF} 
1000 & 0  & 1.346156e-06 & 7.367674e-11 \\ \hline
\rowcolor[HTML]{F1E9FF} 
1000 & 1  & 1.800755e-05 & 9.855748e-10 \\ \hline
\rowcolor[HTML]{F1E9FF} 
1000 & 2  & 1.018017e-04 & 5.571728e-09 \\ \hline
\rowcolor[HTML]{F1E9FF} 
1000 & 3  & 5.166266e-04 & 2.827560e-08 \\ \hline
\rowcolor[HTML]{F1E9FF} 
1000 & 4  & 1.454307e-02 & 7.959598e-07 \\ \hline
\rowcolor[HTML]{F1E9FF} 
1000 & 5  & 9.213756e-02 & 5.042800e-06 \\ \hline
\rowcolor[HTML]{F1E9FF} 
1000 & 6  & 1.780244e+00 & 9.743490e-05 \\ \hline
\rowcolor[HTML]{F1E9FF} 
1000 & 7  & 1.352157e+01 & 7.400518e-04 \\ \hline
\rowcolor[HTML]{F1E9FF} 
1000 & 8  & 6.053637e+01 & 3.313229e-03 \\ \hline
\rowcolor[HTML]{F1E9FF} 
1000 & 9  & 1.527310e+03 & 8.359154e-02 \\ \hline
\rowcolor[HTML]{F1E9FF} 
1000 & 10 & 6.863735e+03 & 3.756605e-01 \\ \hline
\end{tabular}
\end{table}

\clearpage

\begin{table}[]
\centering
\caption{Метод Гаусса с выбором ведущего элемента на матрицах Гильберта}
\begin{tabular}{|c|c|c|}
\hline
\rowcolor[HTML]{EED4E3} 
n     & $\|x^* - x_k\|$            & $\dfrac{\|x^* - x_k\|}{\|x^*\|}$           \\ \hline
\rowcolor[HTML]{F5E5E7} 
3    & 2.625108e-14                                              & 7.015897e-15 \\ \hline
\rowcolor[HTML]{F5E5E7} 
4    & 8.830921e-13                                              & 1.612298e-13 \\ \hline
\rowcolor[HTML]{F5E5E7} 
5    & 3.558406e-11                                              & 4.798154e-12 \\ \hline
\rowcolor[HTML]{F5E5E7} 
6    & 2.319962e-11                                              & 2.431981e-12 \\ \hline
\rowcolor[HTML]{F5E5E7} 
7    & 2.009583e-08                                              & 1.698407e-09 \\ \hline
\rowcolor[HTML]{F5E5E7} 
8    & 1.333407e-06                                              & 9.335716e-08 \\ \hline
\rowcolor[HTML]{F5E5E7} 
9    & 7.270362e-05                                              & 4.306591e-06 \\ \hline
\rowcolor[HTML]{F5E5E7} 
10   & 3.503464e-03                                              & 1.785531e-04 \\ \hline
\rowcolor[HTML]{F5E5E7} 
20   & 8.003411e+02                                              & 1.493943e+01 \\ \hline
\rowcolor[HTML]{F5E5E7} 
30   & 1.081093e+03                                              & 1.111814e+01 \\ \hline
\rowcolor[HTML]{F5E5E7} 
40   & 5.560919e+03                                              & 3.737299e+01 \\ \hline
\rowcolor[HTML]{F5E5E7} 
50   & 2.138525e+04                                              & 1.032189e+02 \\ \hline
\rowcolor[HTML]{F5E5E7} 
100  & 6.160375e+04                                              & 1.059069e+02 \\ \hline
\rowcolor[HTML]{F5E5E7} 
1000 & 3.386377e+08                                              & 1.853405e+04 \\ \hline
\end{tabular}
\end{table}

Аналогично предыдущему методу, для матриц Гильберта приводит к очень большим ошибкам даже при небольшой размерности матриц.

\clearpage


\begin{table}[]
\centering
\caption{МСГ на плотных матрицах Гильберта}
\begin{tabular}{|c|c|c|c|c|}
\hline
\rowcolor[HTML]{C6FCFF} 
n    & \begin{tabular}[c]{@{}c@{}}кол-во\\ итераций\end{tabular} & $\|x^* - x_k\|$            & $\dfrac{\|x^* - x_k\|}{\|x^*\|}$            & cond(A)      \\ \hline
\rowcolor[HTML]{EBFCFF} 
10   & 5                                                         & 3.178324e-01 & 1.619824e-02 & 2.032544e+05 \\ \hline
\rowcolor[HTML]{EBFCFF} 
20   & 6                                                         & 7.960770e-01 & 1.485984e-02 & 1.995758e+05 \\ \hline
\rowcolor[HTML]{EBFCFF} 
30   & 8                                                         & 8.950385e-01 & 9.204728e-03 & 5.751244e+05 \\ \hline
\rowcolor[HTML]{EBFCFF} 
40   & 8                                                         & 2.043063e+00 & 1.373071e-02 & 2.123930e+05 \\ \hline
\rowcolor[HTML]{EBFCFF} 
50   & 10                                                        & 1.455476e+00 & 7.025059e-03 & 9.638469e+05 \\ \hline
\rowcolor[HTML]{EBFCFF} 
100  & 12                                                        & 4.095343e+00 & 7.040560e-03 & 7.703249e+05 \\ \hline
\rowcolor[HTML]{EBFCFF} 
200  & 14                                                        & 1.137471e+01 & 6.939539e-03 & 6.666093e+05 \\ \hline
\rowcolor[HTML]{EBFCFF} 
300  & 14                                                        & 3.060519e+01 & 1.017629e-02 & 2.860521e+05 \\ \hline
\rowcolor[HTML]{EBFCFF} 
400  & 14                                                        & 5.966772e+01 & 1.289427e-02 & 1.526146e+05 \\ \hline
\rowcolor[HTML]{EBFCFF} 
500  & 17                                                        & 5.331629e+01 & 8.247355e-03 & 4.310517e+05 \\ \hline
\rowcolor[HTML]{EBFCFF} 
1000 & 18                                                        & 1.440971e+02 & 7.886610e-03 & 1.205901e+05 \\ \hline
\end{tabular}
\end{table}

Заметим, что в отличие от методов Гаусса, МСГ абсолютная и относительная погрешность существенно меньше, а абсолютная погрешность линейно зависит от размерности.

\clearpage

\begin{table}[]
\centering
\caption{МСГ на матрице с диагональным преобладанием}
\begin{tabular}{|c|c|c|c|c|}
\hline
\rowcolor[HTML]{BCB8CE} 
n    & \begin{tabular}[c]{@{}c@{}}кол-во\\ итераций\end{tabular} & $\|x^* - x_k\|$            & $\dfrac{\|x^* - x_k\|}{\|x^*\|}$            & cond(A)      \\ \hline
\rowcolor[HTML]{D5DDEF} 
10   & 9                                                         & 5.404132e-14 & 2.754201e-15 & 4.166323e+00 \\ \hline
\rowcolor[HTML]{D5DDEF} 
100  & 12                                                        & 1.403758e-05 & 2.413287e-08 & 4.986249e-01 \\ \hline
\rowcolor[HTML]{D5DDEF} 
1000 & 10                                                        & 1.999675e-04 & 1.094446e-08 & 5.053610e-01 \\ \hline
\rowcolor[HTML]{D5DDEF} 
10000 & 67                                                        & 5.213623e-02 & 3.257148-07 & 3.490549e+00 \\ \hline
\rowcolor[HTML]{D5DDEF} 
100000 & 72                                                        & 2.073174e+00 & 1.135515-07 & 1.752933e+00 \\ \hline
\end{tabular}
\end{table}


\begin{table}[]
\centering
\caption{МСГ на матрице с обратным знаком внедиагональных элементов}
\begin{tabular}{|c|c|c|c|c|}
\hline
\rowcolor[HTML]{BA7735} 
n    & \begin{tabular}[c]{@{}c@{}}кол-во\\ итераций\end{tabular} & $\|x^* - x_k\|$            & $\dfrac{\|x^* - x_k\|}{\|x^*\|}$            & cond(A)      \\ \hline
\rowcolor[HTML]{EFD091} 
10   & 9                                                         & 1.601063e-12 & 8.159773e-14 & 6.672343e+01 \\ \hline
\rowcolor[HTML]{EFD091} 
100  & 12                                                        & 7.113634e-06 & 1.222949e-08 & 5.021166e-01 \\ \hline
\rowcolor[HTML]{EFD091} 
1000 & 10                                                        & 1.654867e-04 & 9.057290e-09 & 4.988307e-01 \\ \hline
\rowcolor[HTML]{EFD091} 
10000 & 54                                                        & 1.958591e-01 & 9.124046-07 & 1.784023+00 \\ \hline
\rowcolor[HTML]{EFD091} 
100000 & 81                                                        & 3.197423e+00 & 9.328380e-07 & 1.784023e+00 \\ \hline
\end{tabular}
\end{table}

\clearpage

\section{Выводы}

Из таблиц можно было сделать вывод, что порядок погрешностей методов Гаусса совпадают.

Если сравнивать методы на матрицах Гильберта, можно заметить, что при использовании МСГ погрешности существенно меньше, а при использовании  методов Гаусса мы видим экспоненциальный рост относительной погрешности.

Минус МСГ для решения СЛАУ - применимо только для симметричных матриц, но при этом, в случае хорошо обусловленных матриц за небольшое количество итераций, из чего следует, что на матрицах произвольно обусловленных, количество итераций получается порядка квадрата размерности.

При том, что порядок погрешностей методов Гаусса совпадает, прямой ход с выбором ведущего элемента работает за $\mathcal{O}(n^2) + \dfrac{n^3}{3}$, a с предварительно посчитанным LU-разложением — за $\mathcal{O}(n^2)$, получается, что второй алгоритм эффективнее для фиксированной матрицей.

\end{document}